% Save this as tutorial.tex for the lwarp package tutorial.
\documentclass{book}
\usepackage{iftex}

% --- LOAD FONT SELECTION AND ENCODING BEFORE LOADING LWARP ---
\ifPDFTeX
	\usepackage{lmodern} % pdflatex or dvi latex
	\usepackage[T1]{fontenc}
	\usepackage[utf8]{inputenc}
\else
	\usepackage{fontspec} % XeLaTeX or LuaLaTeX
\fi

% --- LWARP IS LOADED NEXT ---

\usepackage[
% HomeHTMLFilename=index, % Filename of the homepage.
% HTMLFilename={node-}, % Filename prefix of other pages.
% IndexLanguage=english, % Language for xindy index, glossary.
% latexmk, % Use latexmk to compile.
% OSWindows, % Force Windows. (Usually automatic.)
mathjax, % Use MathJax to display math.
]{lwarp}

% \boolfalse{FileSectionNames} % If false, numbers the files.

% --- LOAD PDFLATEX MATH FONTS HERE ---

% --- OTHER PACKAGES ARE LOADED AFTER LWARP ---

\usepackage{standalone}
\usepackage{tikz}
\usetikzlibrary{positioning}

\usepackage{makeidx} \makeindex
\usepackage{xcolor}               % (Demonstration purposes only.)
\usepackage{hyperref,cleveref}    % LOAD THESE LAST!

% --- LATEX AND HTML CUSTOMIZATION ---
\title{Lwarp Reference}
\author{Me}
\setcounter{tocdepth}{2} 					% Include subsections in the \TOC.
\setcounter{secnumdepth}{2} 			% Number down to subsections.
\setcounter{FileDepth}{1} 				% Split \HTML\ files at sections
\booltrue{CombineHigherDepths} 		% Combine parts/chapters/sections
\setcounter{SideTOCDepth}{1} 			% Include subsections in the side\TOC

% HTML Directives
\HTMLTitle{Webpage Title} 				% Overrides \title for the web page.
\HTMLAuthor{Some Author} 					% Sets the HTML meta author tag.
\HTMLLanguage{en-US} 							% Sets the HTML meta language.
\HTMLDescription{A description.} 	% Sets the HTML meta description.
\HTMLFirstPageTop{Name and \fbox{HOMEPAGE LOGO}}
\HTMLPageTop{\fbox{LOGO}}
\HTMLPageBottom{Contact Information and Copyright}
\CSSFilename{lwarp_formal.css}
% \CSSFilename{lwarp_sagebrush.css}

\begin{document}

\maketitle % Or titlepage/titlingpage environment.
% An article abstract would go here.

\tableofcontents % MUST BE BEFORE THE FIRST SECTION BREAK!
\listoffigures

\chapter{First chapter}
\section{A section}

This is some text which is indexed.\index{Some text.}

\subsection{A subsection}

See \cref{fig:withtext}.

\begin{figure}
	\begin{center}
		\fbox{\textcolor{blue!50!green}{Text in a figure.}}
		\caption{A figure with text\label{fig:withtext}}
	\end{center}
\end{figure}

\section{Some math}

Inline math: $r = r_0 + vt - \frac{1}{2}at^2$
followed by display math:

\begin{equation}
a^2 + b^2 = c^2
\end{equation}

\documentclass[class=article, crop=false]{standalone}

\begin{document}

This submodule simply includes a tikz drawing which is itself a standalone file

% Import paths relative to main document
\documentclass{standalone}

\begin{document}

Will this import???

\begin{tikzpicture}[
vertex/.style={circle, draw, fill=white, thick, minimum size=20},
]

% Vertices
\node[vertex] (S) {$S$};
\node[vertex] (V) [above right=of S] {$V$};
\node[vertex] (W) [below right=of S] {$W$};
\node[vertex] (T) [below right=of V] {$T$};

% Edges
\draw[->, thick] (S) to node[above] {1} (V);
\draw[->, thick] (S) to node[below] {4} (W);
\draw[->, thick] (V) to node[right] {2} (W);
\draw[->, thick] (V) to node[above] {6} (T);
\draw[->, thick] (W) to node[below] {3} (T);

\end{tikzpicture}

\end{document}


\end{document}


\begin{warpprint} % For print output ...
	\cleardoublepage % ... a common method to place index entry into TOC.
	\phantomsection
	\addcontentsline{toc}{chapter}{\indexname}
\end{warpprint}

\ForceHTMLPage 	% HTML index will be on its own page.
\ForceHTMLTOC 	% HTML index will have its own toc entry.
\printindex
\end{document}
